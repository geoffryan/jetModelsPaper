\documentclass[twocolumn]{aastex62}

\usepackage{amsmath}
\usepackage{amssymb}
\usepackage{amsfonts}
\usepackage{graphicx}

\newcommand{\eps}{\ensuremath{\epsilon}}
\newcommand{\gam}{\ensuremath{\gamma}}
\newcommand{\Gam}{\ensuremath{\Gamma}}
\newcommand{\Del}{\ensuremath{\Delta}}

\newcommand{\gwbns}{GW170817}
\newcommand{\afterglowpy}{{\tt afterglowpy}}
\newcommand{\python}{\tt{Python}}

\newcommand{\dd}{\ensuremath{\mathrm{d}}}
\newcommand{\Mej}{\ensuremath{M_{\mathrm{ej}}}}
\newcommand{\tobs}{\ensuremath{t_{\mathrm{obs}}}}
\newcommand{\tW}{\ensuremath{t_{\mathrm{w}}}}
\newcommand{\tb}{\ensuremath{t_{\mathrm{b}}}}
\newcommand{\tdec}{\ensuremath{t_{\mathrm{dec}}}}
\newcommand{\tumin}{\ensuremath{t_{\mathrm{min}}}}
\newcommand{\nuobs}{\ensuremath{\nu_{\mathrm{obs}}}}
\newcommand{\thobs}{\ensuremath{\theta_{\mathrm{obs}}}}
\newcommand{\thW}{\ensuremath{\theta_{\mathrm{w}}}}
\newcommand{\thC}{\ensuremath{\theta_{\mathrm{c}}}}
\newcommand{\epse}{\ensuremath{\varepsilon_{\mathrm{e}}}}
\newcommand{\epseb}{\ensuremath{\bar{\varepsilon}_{\mathrm{e}}}}
\newcommand{\epsB}{\ensuremath{\varepsilon_{\mathrm{B}}}}
\newcommand{\sigT}{\ensuremath{\sigma_{\mathrm{T}}}}
\newcommand{\dL}{\ensuremath{d_{\mathrm{L}}}}
\newcommand{\qe}{\ensuremath{q_{\mathrm{e}}}}
\newcommand{\Mp}{\ensuremath{m_{\mathrm{p}}}}
\newcommand{\Me}{\ensuremath{m_{\mathrm{e}}}}
\newcommand{\Eiso}{\ensuremath{E_{\mathrm{iso}}}}
\newcommand{\geff}{\ensuremath{g_{\mathrm{eff}}}}
\newcommand{\umin}{\ensuremath{u_{\mathrm{min}}}}
\newcommand{\umax}{\ensuremath{u_{\mathrm{max}}}}




\begin{document}

\title{GRB Afterglows In The Multi-Messenger Era}
\correspondingauthor{Geoffrey Ryan}
\email{gsryan@umd.edu}

\author[0000-0001-9068-7157]{Geoffrey Ryan}
\affil{Joint Space-Science Institute, University of Maryland, College Park, MD 20742, USA}

\author{Hendrik van Eerten}


\begin{abstract}
	Abstract abstract abstract
\end{abstract}

\section{outline}

\begin{itemize}
	\item Need to consider the \emph{structure} of GRB explosions (esp. binary neutron stars).  Structure may be radial (velocity stratification, $E(u)$) or angular ($E(\theta)$).  Both modes affect the afterglow.
	\item Analytic description: \begin{itemize}
		\item Velocity stratification / refreshed shock / cocoon leads to simple power-law behaviour depending on $k$ ($E_{>u} \propto u^{-k}$).  Breaks occur at deceleration time and when $E_{>u}$ shuts off.
		\item Angular structure / structured jet leads to not-quite power-law behaviour, depending on $g \approx (\thobs - \theta)d \log E / d\theta$.  Breaks occur when $\thobs - \thW \approx 1/\gamma(\thW)$ and $\thobs \approx 1/\gamma(0)$.  $g = g(\theta)$, but $g\approx g_{eff}$ where $g_{eff} = g(\thobs/2)$.
		\end{itemize}
	\item Numerical description: \begin{itemize}
		\item Single-shell ODE for shock evolution
		\item Integrate over equal-$\tobs$ surfaces
		\item Structured components independent.
		\item public and open: install with \tt{pip}.
		\end{itemize}
	\item GW170817A - Both models are similar, fit GW170817A well!
	\item Future observations: Rate of observed counter-parts will depend on degree of collimation of afterglow (jets have lower observed rates than cocoons).  Radial structure may have $k$ vary from object to object.  Angular structure WILL have $g$ vary from object to object.
	\item Can we determine structure from the afterglow???
\end{itemize}

\section{Introduction}

GRB afterglows are commonly modeled as top hat blast waves with no internal structure pointed directly at the observer.  This has been largely sufficient given the data.  (Although: viewing angle effects can be measured (Ryan+ 2015)).  New observational capabilities have witnessed new phases of afterglow evolution, such as the discovery of plateaus with Swift.  Real jets (of course) have internal structure, but the top hat assumption has been a sufficiently spherical cow. 

GW170817 changes the picture: new object with a strange afterglow.  Steep/late rise, LONG slow rise, turnover at 160d.  The slow rise is impossible to reproduce with standard afterglow evolution: first observation of a \emph{structured} phase of evolution. 

The dominant contribution may be an angular distribution of energy within the blast wave (``structured jet'') combined with off-axis viewing (CITE Rossi04).  For 170817 this corresponds to the ``successful jet'' picture.  Alternately, radial stratification of velocity (``refreshed shock'') can cause a similar phase, most simply with a quasi-spherical (or at least top-hat) angular structure.  This situation is produced by the ``choked jet''  or ``cocoon'' picture of 170817. Understanding the effects of both structures on afterglows, and their dependence on the parameters of the burst, are vital to understanding GW counterparts.

To this end, we have calculated the power-law temporal slope of the structured phase in both angular and radial structure dominated cases, along with the corresponding observed break times.  We have developed a semi-analytic numerical model of structured GRB afterglows for detailed comparison to data, which is now publicly available.

Summary: In section 2 we discuss structured jets, in section 3 refreshed shocks, in section 4 our numerical model, section 5 application to GW170817, and section 6 future applications and discussion.

\section{Structured Jets}\label{sec:structuredJets}

The \emph{structured jet} is a model where the isotropic-equivalent energy of a blast wave is a function of the angle from the jet axis: $\Eiso = E(\theta)$. The specific structure of a jet is determined by the intrinsic structure of the jet launching mechanism as well as the sculpting that occurs as the jet burrows out of the encasing ejecta debris (as in a binary neutron star merger) or stellar envelope (as in a collapsar).  

Numerical simulations have revealed a variety of jet angular energy distributions, often containing an energetic core with power-law tails.  Figure \ref{fig:jetStruct} shows a collection of jet energy distributions from the literature (CITE curves in plots).  Lacking a well-established physical model of the true $E(\theta)$, in particular its dependence on the parameters of the system, we consider two simple parameterized models: a Gaussian jet and a power-law.  

\begin{align}
	E(\theta) &= E_0 \exp\left(-\frac{\theta^2}{2\thC^2}\right)  && \text{Gaussian} \label{eq:gauss}\\
	E(\theta) &= \frac{E_0}{1 + (\theta/ \thC)^2}  && \text{power-law} \label{eq:pl}
\end{align}

\begin{figure}
	\plotone{figs/jetStruct.pdf}
	\caption{Jet Structure.  Temporary version.  Remove $\Gamma$ subplot.  Add Aloy. \label{fig:jetStruct}}
\end{figure}

Each model is parameterized by a normalization $E_0$, a width $\thC$, and a truncation angle $\thW$.  Outside $\thW$ the energy is initially zero.  Figure \ref{fig:jetStruct} shows an example of each model fit to recent simulation results.  The Gaussian model emulates jets with sharp edges, while the power-law emulates jets with narrow cores and large wings (AWK. jargon?). 

To observers, an energy gradient along the blast wave surface presents itself as a slow, rising or decaying, evolution of the flux.  The usual (pre-jet break) decay is compensated for by regions of higher energy continuously coming into view.  These new contributions to the emission slow the decay and can even cause the observed flux to increase with time.  

We call this the \emph{structured} phase of emission.  It begins at observer time $\tW$, once the blast wave is decelerating and the observer is within the beaming cone of any part of the emitting surface.  It ends at observer time $\tb$, when the jet core has become fully visible.  These times are:
\begin{align}
	\tW &= \left(\frac{9}{16\pi} \frac{E(\thW)}{n_0 \Mp c^5}\right)^{1/3} \left( 2 \sin \left(\frac{\thobs-\thW}{2}\right)\right)^{8/3} \ ,\label{eq:tw} \\
	\tb &= \left(\frac{9}{16\pi} \frac{E_0}{n_0 \Mp c^5}\right)^{1/3} \left( 2 \sin \left(\frac{\thobs}{2}\right)\right)^{8/3} \ .\label{eq:tb}
\end{align}

\begin{figure}
	\plotone{figs/jetDecomp.pdf}
	\caption{A structured jet, decomposed into emission from different latitudes $\theta$.  The structured phase is the result of the brightest point of the blast-wave tracking from the wings to the jet core. \label{fig:decomp}}
\end{figure}
Figure \ref{fig:decomp} shows the light curve of a structured jet along with the emissions from different latitudes.  The part of the blast wave dominating the emission tracks from the high latitude wings $\theta \approx \thW$ to the core $\theta \approx 0$.

During the structured phase, the afterglow has a temporal spectral slope which depends on the parameter $g(\theta)$:
\begin{equation}
	g(\theta) \equiv -2 \tan\left(\frac{\thobs - \theta}{2}\right)\frac{d \log E}{d \theta}\ . \label{eq:gdef}
\end{equation}

The temporal and spectral slopes for various regimes are given in Table \ref{tab:slopes}.  The derivation of the slopes, $\tW$, and $\tb$ are given in Appendix \ref{sec:derive1}.

\begin{deluxetable*}{CCCCCC}
	\tablecaption{Structured jet temporal and spectral slopes: $F_\nu \propto \tobs^\alpha \nu^\beta$. See Equations \eqref{eq:gdef} and \eqref{eq:geff} for the definition and effective values of $g$.  For a refreshed-shock model replace $g$ with $k$, defined in Equation \eqref{eq:Eu}. \label{tab:slopes}}
	\tablehead{\colhead{Regime}& \colhead{Label} & \colhead{$\alpha_{\mathrm{off-axis}}$}& \colhead{$\alpha_{\mathrm{struct}}$}& \colhead{$\alpha_{\mathrm{struct,max}}$} & \colhead{$\beta$}}
	\startdata
	\nu<\nu_m<\nu_c     & D & 7 			& \frac{-2+3s_3 + 3g}{8+g} 		& 3		& 1/3 \\
	\nu_m<\nu<\nu_c     & G & 15/2 - 3p/2 	& \frac{- 6p + 3s_3 + 3g}{8+g} 		& 3		& (1-p)/2 \\
	\nu < \nu_c < \nu_m & E & 17/3 		& \frac{-14/3 + 3s_3 + 11g/3}{8+g} 	& 11/3	& 1/3 \\
	\nu_c < \nu < \nu_m & F & 13/2 		& \frac{-8+3s_3 + 2g}{8+g} 		& 2		& -1/2 \\
	\nu_c < \nu_m < \nu & H & 8 - 3p/2 		& \frac{-6p - 2 + 3s_3 + 2g}{8+g} 	& 2		& -p/2 \\ 
	\enddata
\end{deluxetable*}

The $g$ parameter evolves with time as the brightest sector of the jet sweeps from the edge to the core.  This produces deviations in the light curve from a pure power-law.  However, we find ultimately these deviations are not too large and the average slope is well approximated by $\geff  = g(\thobs/2)$.  For the power-law and Gaussian jet models this gives:
\begin{align}
	\geff &= \frac{\thobs^2}{4\thC^2} && \text{Gaussian}\ , \\
	\geff &= \frac{2 \thobs^2}{4 \thC^2+\thobs^2} && \text{power-law}\ . \label{eq:geff}
\end{align}
These values may be used in Table \ref{tab:slopes} for an approximate power-law model of a structured jet with an appropriate viewing angle dependent temporal evolution.  These analytic results are in agreement with previous work on structured jets (CITE ROSSI) and the trans-relativistic numerical model described in Section \ref{sec:numerical}.

In the structured phase, the temporal slope depends directly on the ratio $\thobs/\thC$.  Therefore any measurement of the effective power law slope of an afterglow light curve can be used to measure this parameter.

\section{Refreshed Shocks}\label{sec:refreshedShocks}

An alternative mechanism to produce slow decays or rises in afterglow light curves is the \emph{refreshed shock}, where velocity stratification of the ejecta causes a prolonged period of energy injection. Slow material initially coasting behind the shock is gradually incorporated into the blast wave as it decelerates, ``refreshing'' the shock.  This mechanism was first proposed as an energy injection scenario (CITE), and recently has been proposed as the mechanism behind GW170817 in the choked-jet scenario. (CITE Mooley, Hotokezaka, etc).

In the simplest case the visible part of the blast wave is assumed to be quasi-spherical.  The distribution of material behind the blast wave is specified by $E_{>}(u)$, the energy of all material in the ejecta with four-velocity greater than $u$.  This is typically taken to be a power-law in $u$ within the finite domain $[\umin, \umax]$:
\begin{equation}
	E_>(u) = E_r u^{-k}\ , \label{eq:Eu}
\end{equation}
where $u \in (\umin, \umax)$ is the dimensionless four-velocity, $k>0$ is the power law index, and $E_r$ is a normalization factor.  

A blast wave will have an initial coasting period before sweeping up enough mass in the ambient medium to begin deceleration and be subject to refreshed shocks.  The shock refreshment ends when the blast wave decelerates past $\umin$.  These times are given by:
\begin{align}
	\tdec &= ... \ ,\label{eq:tdec} \\
	\tumin &= \left(\frac{9}{16\pi} \frac{E_r}{n_0 \Mp c^5}\right)^{1/3} \umin^{-8/3} \ .\label{eq:tumin}
\end{align}
The temporal and spectral slopes in this mechanism are identical to the structured jet, but with the replacement $k \leftrightarrow g$.  For a derivation see Appendix \ref{sec:derive3}.

\section{Numerical Model: The \afterglowpy{} Package}\label{sec:numerical}

To examine the detailed behavior of structured afterglows we constructed a numerical model, publicly available as the \afterglowpy{} Python package.

The numerical model employs a trans-relativistic equation of state to evolve a blast wave in the single-shell approximation from the initial coasting phase through deceleration and spreading into the non-relativistic Sedov phase.  The observed flux density $F_\nu$ is calculated by numerically integrating the synchrotron emissivity over equal-observer-time surfaces.  

The flux calculation follows the semi-analytic method of (CITE Hendrik 2010) using the shock evolution prescription outlined in (CITE Hendrik Review) modified to include refreshed shocks, energy injection, and jet spreading.  For a structured jet, the surface is divided into $N_\theta$ annular regions each of which is allocated an energy according to $E(\theta)$.  Each segment evolves independently and is otherwise treated as a section of a top-hat jet.

\subsection{Numerical Algorithm}\label{subsec:algo}

To compute the light curve of a top-hat jet we first must calculate the time evolution of the blast wave.  We utilize the single-shell approximation, treating the ejecta mass, contact discontinuity, and forward shock as a single unit propagating through a cold ambient medium with constant rest-mass density $\rho_0 = \Mp n_0$.  For a detailed derivation see Appendix \ref{sec:deriveNum} and (CITE Hendrik 2010 + Hendrik Review).  The forward shock is a radius $R$ from the explosion and moves at a velocity $\beta_s$.  In terms of the fluid four-velocity behind the shock $u$ and its Lorentz factor $\gamma$ this may be written as:
\begin{equation}
	\dot{R} = \frac{4 u \gamma}{4 u^2 +3}c\ . \label{eq:Rdot}
\end{equation}
Here the time derivative $\dot{R}$ is taken with respect to elapsed time in the bursters' frame $t$.  

 The evolution of the four velocity is determined through conservation of energy.  The total energy in the trans-relativistic single shell approximation is:
\begin{equation}
	E = (\gamma - 1)\Mej c^2 + \frac{4\pi}{9} \rho_0 c^2 R^3 (4 u^2 + 3) \beta^2 \Omega\ . \label{eq:E}
\end{equation}
In Equation \eqref{eq:E} the first term is the kinetic energy of the ejected mass $\Mej$, assumed to have already accelerated and adiabatically cooled to its coasting velocity. The second term is the kinetic and thermal energy of the shocked ISM with three-velocity $\beta = u / \gamma$. The shocked ISM is confined to a cone of solid angle $\Omega = 2 \sin^2 (\theta_0/2)$ where $\theta_0$ is the half-opening angle of the jet. While the blast wave is relativistic $\theta_0$ is constant, but once the jet becomes non-relativistic it begins to spread due to its own internal pressure until the blast wave is spherical.  The spreading occurs at sound speed $c_s$ in the fluid rest-frame (CITE Hendrik 2010, Duffell+Laskar 2017):
\begin{equation}
	\dot{\theta_0} = \frac{c_s}{\gamma R}\ , \qquad \text{if } u < 1\text{ and } \theta_0 > \pi/2\ . \label{eq:thetadot}
\end{equation}
Given a (possibly constant) expression for $E$ Equation \eqref{eq:E} may be differentiated with respect to $t$ and solved for $\dot{u}$.  In a refreshed shock scenario $E = E_{>}(u)$, whereas for a standard energy injection model one may take $E \propto t^{-q}$.  In either case, Equation \eqref{eq:E} produces an equation for $\dot{u}$.  This defines a three dimensional system of ordinary differential equations in the variables $(R, u, \theta_0)$.  We solve this system using RK4 on a sequence of geometrically spaced values of $t$, typically $10^3$ points per decade.  This produces a numerical table of $R(t)$, $u(t)$, and $\theta_0(t)$ in terms of the burster time $t$.  The beginning and end points of the solution are chosen to encompass all required values of $\tobs$.

Once the shock evolution is known, the flux is given by the integral:
\begin{equation}
	F_\nu(\tobs, \nuobs) = \frac{1}{4\pi \dL^2} \int \! \dd \Omega\ R^2\ \Delta R\  \delta^2\ \epsilon'_{\nu'} \ , \label{eq:flux}
\end{equation}
where $\tobs$ is the time in the observer's frame, $\nuobs$ the observational frequency, $\dL$ the luminosity distance, $\Delta R$ the shock width, $\delta$ the Doppler factor of the fluid, and $\epsilon'_{\nu'}$ the synchrotron emissivity in the fluid rest frame.  The integral is taken in spherical polar coordinates $(\theta, \phi$) where $\theta = 0$ aligns with the jet axis and $\phi=0$ is directed towards the observer.  The observer is at an angle $\thobs$ with the jet axis. The integrand is evaluated at a constant observer time $\tobs$, related to $t$ by:
\begin{equation}
	\tobs = t - \mu(\theta, \phi) R(t)/c\ , \label{eq:tobs}
\end{equation}
where 
\begin{equation}
	\mu = \cos \theta \cos \thobs + \sin \theta \sin \thobs \cos \phi\ . \label{eq:mu}
\end{equation}
The integral \eqref{eq:flux} is evaluated numerically using an adaptive Romberg integrator in each dimension, with a typical relative tolerances of $10^{-6}$.  Each evaluation of the integrand requires a binary search to determine the burster time $t$ at which to evaluate $R$, $u$, and $\theta_0$.  Fluid quantities are then calculated using the shock-jump conditions and the synchrotron emissivity is evaluated using the standard external shock formulae (CITE Hendrik, Granot+Sari).

The numerical accuracy of a top-hat light curve is typically better than $10^{-4}$.  The structured jet calculation splits the integration domain into $N_\theta$ disjoint annuli, each evaluated as an independent top-hat and summed.  By default $N_\theta$ is chosen such that there will be 5 zones per $\thC$-sized interval, giving final light curves with a numerical accuracy of $10^{-2}$.

\subsection{\afterglowpy{}}

We have constructed the \afterglowpy{} Python package to implement this algorithm and provide it to the community.  The integration routine itself is written in C, wrapped as an extension for Python, and has been optimized to be used in intensive data analysis routines such as Markov Chain Monte-carlo which can require many thousands of evaluations.  

\afterglowpy{} is available on PyPI and may be installed with {\tt pip}.  The source code is open and available at https://github.com/geoffryan/afterglowpy.

\section{Application: GW170817A}

\section{Discussion}


\newpage

\section{Structured Phase}

\begin{figure}
	\plottwo{figs/slopesJet.png}{figs/slopesCocoon.png}
	\caption{Temporal slopes for the jet (left) and cocoon (right).}
\end{figure}








\appendix
\section{Derivation of the Off-Axis Jet equations}\label{sec:derive1}

  The \emph{structured jet} model is a generalization of the simple top hat jet where the energy and Lorentz factor vary with the polar angle.  The light curves of structured jets display more complex behavior than top hats, which we can understand through some simple analytic relationships.
  
  Firstly, the complex behavior of a structured jet is due to relativistic beaming enhancing the jet emission at different angles as a function of time.  Once the jet becomes non-relativistic this effect is suppressed and the entire jet comes into view.  As such we will focus on the emission when the jet remains relativistic, the late-time behavior is the same as any Newtonian jet of comparable total energy.  Numerical simulations and analytic considerations have demonstrated that jet spreading does not begin in earnest until the blast wave approaches sub relativistic velocity so we will also neglect the effects of spreading and assume each sector of the jet evolves independently.  Lastly, we will assume when each sector of the blast wave is visible it is in the deceleration regime.  In this phase of evolution the blast wave Lorentz factor evolves according to:
  \begin{equation}
	\gamma(t; \theta) \propto \sqrt{\frac{E(\theta)}{n_0}}\ t^{-3/2}\ . \label{eq:lorentzEvolution}
\end{equation}
  In the above $\gamma$ is the Lorentz factor of the shocked fluid at an angle $\theta$ from the jet axis at lab time (in the burster frame) $t$.  The blast wave expands into a medium of constant density $n_0$ and has an angularly dependent isotropic-equivalent energy $E(\theta)$.  The forward shock is at a position $R(t; \theta)$, and moves at a speed $\beta_s = (1-\gamma_s^{-2})^{1/2}$, where $\gamma_s^2 = 2 \gamma^2$ is the Lorentz factor of the shock.  Assuming $\gamma \gg 1$ gives:
\begin{equation}
	R(t; \theta) = ct\left(1-\frac{1}{16 \gamma^2(t; \theta)}\right)\ .
\end{equation}
We denote the angle between the viewer and a particular jet sector as $\psi$ and its cosine and $\mu = \cos \psi$.  Photons emitted at time from a sector of the blast wave at time $t$ will be seen by the observer at $\tobs$:
\begin{equation}
	\tobs = t - \frac{\mu}{c} R(t,\theta) = (1-\mu)t + \frac{\mu}{16}\frac{t}{ \gamma^2(t,\theta)} \label{eq:tobs}
\end{equation}

The observed flux depends on the luminosity distance $d_L$, viewing angle $\thobs$, and rest-frame emissivity $\varepsilon'_{\nu'}$.  The Doppler factor is $\delta = \gamma^{-1} (1-\beta\mu)^{-1}$, where $\beta=(1-\gamma^{-2})^{1/2}$ is the fluid three velocity.  The observed flux can then be expressed as a volume integral, where the integrand is evaluated at the time $t$ corresponding to $\tobs$ and position ${\bf r}$.
\begin{equation}
	F_\nu(\tobs, \nuobs) = \frac{1}{4\pi d_L^2} \int d^3{\bf r} \ \delta^2 \varepsilon'_{\nu'}\ .
\end{equation}  
The blast wave emits from a region of width $\Delta R \propto \delta_s \gamma_s \gamma^{-2} R $ where $\delta_s$ is the Doppler factor associated with the shock Lorentz factor $\gamma_s$. At a given observer time, the emission will be dominated by a region of (rest-frame) angular size $\Delta \Omega$.  The flux can then be approximated as:
\begin{equation}
	F_\nu(\tobs, \nuobs) \propto R^2 \Delta R \Delta \Omega \delta^2 \varepsilon'_{\nu'} \propto \Delta \Omega\ t^3 \gamma^{-2} \gamma_s \delta_s \delta^2 \varepsilon'_{\nu'}
\end{equation}.
The emissivity $\varepsilon'_{\nu'}$ depends on the fluid Lorentz factor, the frequency $\nuobs$, and numerous (constant) microphysical parameters.  We parameterize the dynamic dependence as $\varepsilon'_{\nu'} \propto \gamma^{s_1} t^{s_2} {\nu'}^\beta = t^{s_2} \gamma^{s_1}\delta^{-\beta} \nuobs^\beta$. The values of $s_1$, $s_2$, and $\beta$ in synchrotron regimes are given in Table \ref{tab:specSlopes}.  This leads to a flux of:
\begin{equation}
	F_\nu \propto \Delta \Omega\ t^{3+s_2} \gamma^{-1+s_1} \delta_s \delta^{2-\beta} \nuobs^\beta\ . \label{eq:fluxApprox}
\end{equation}

\begin{deluxetable}{CCCC}
	\tablecaption{Dependence of $\varepsilon'_{\nu'} \propto \gamma^{s_1} t^{s_2} \nu'^{\beta}$ in various spectral regimes. \label{tab:specSlopes}}
	\tablehead{\colhead{Regime}& \colhead{$s_1$} & \colhead{$s_2$} & \colhead{$\beta$}}
	\startdata
	\nu'<\nu'_m<\nu'_c     & 1 & 0 & 1/3 \\
	\nu'_m<\nu'<\nu'_c     & (3p+1)/2 & 0 & (1-p)/2 \\
	\nu'_m<\nu'_c<\nu'     & 3p/2 & -1 & -p/2 \\ 
	\hline 
	\nu' < \nu'_c < \nu'_m & 7/3 & 2/3 & 1/3 \\
	\nu'_c < \nu' < \nu'_m & 3/2 & -1 & -1/2 \\
	\nu'_c < \nu'_m < \nu' & 3p/2 & -1 & -p/2 \\ 
	\enddata
\end{deluxetable}
The Doppler factor depends on the fluid Lorentz factor and whether the material is on-axis. The on/off axis boundary occurs at $\gamma^{-1} \approx \sin \psi$ (ie. $\beta \approx \mu$).  One can write:
\begin{equation}
	\delta = \left \{ \begin{matrix}
				\frac{1}{2} \gamma^{-1} \sin^{-2}\psi/2,  & \text{if } \gamma \gg 1 \text{ and } \sin{\psi} \gg \gamma^{-1} \ \ \text{(off-axis)} \\
				2 \gamma, & \text{if } \gamma \gg 1 \text{ and } \sin{\psi} \ll \gamma^{-1} \ \  \text{(on-axis)} \\
				1  & \text{if } \beta \ll 1\ \  \text{(non-relativistic)} \\ \end{matrix} \right . \ .
\end{equation}
The observer time can be similarly simplified in both limits:
\begin{equation}
	\tobs = \left \{ \begin{matrix}
				\frac{1}{2} \psi^2 t,  & \text{if }\sin{\psi} \gg \gamma^{-1}\ \ \text{(off-axis)} \\
				\frac{1}{16} \gamma^{-2} t, & \text{if } \sin{\psi} \ll \gamma^{-1} \ \  \text{(on-axis)} \end{matrix} \right . \ .
\end{equation}
As can the flux:
\begin{equation}
	F_\nu \propto \left \{ \begin{matrix}
				\Delta \Omega t^{3+s_2} \gamma^{-4+s_1+\beta} \psi^{-6+2\beta}\nuobs^\beta,  & \text{(off-axis)} \\
				\Delta \Omega t^{3+s_2} \gamma^{2+s_1-\beta} \nuobs^\beta, & \text{(on-axis)} \end{matrix} \right . \ .
\end{equation}
The behavior of the Doppler factor informs the scaling of $\Delta \Omega$.  If any part of the jet is on-axis, its emission is enhanced by $\sim \gamma^2$ over the off-axis material and will dominate.  An observer for whom the entire jet is off-axis must be situated at some large $\thobs$, outside the outermost jet material.  At early times the entire jet will be beamed off-axis, with emission from the near edge (with the smallest $\psi$ and presumably $\gamma$) contributing most to the emission.  The absence of any particular angular scale in this regime indicates $\Delta \Omega$ will be roughly constant.  For an on-axis observer $\Delta \Omega \sim \sin^2 \psi_{\rm{max}} \propto \gamma^{-2}$ until the entire jet is on-axis, at which point $\Delta \Omega$ is again constant.  Hence:
\begin{equation}
	F_\nu \propto \left \{ \begin{matrix}
				t^{3+s_2} \gamma^{-4+s_1+\beta} \psi^{-6+2\beta}\nuobs^\beta,  & \text{(off-axis)} \\
				t^{3+s_2} \gamma^{s_1-\beta} \nuobs^\beta, &  \text{(on-axis, pre jet break)} \\
				t^{3+s_2} \gamma^{2+s_1-\beta} \nuobs^\beta, & \text{(on-axis, post jet break)} \end{matrix} \right . \ .
\end{equation}
Finally, for off-axis emission $t \propto \tobs$ and hence $\gamma \propto \tobs^{-3/2}$.  For on-axis observers $\tobs \propto \gamma{-2} t \propto t^4$, hence $t\propto \tobs^{1/4}$ and $\gamma \propto \tobs^{-3/8}$. Giving finally:
\begin{equation}
	F_\nu \propto \left \{ \begin{matrix}
				\tobs^{9+s_2 -3(s_1+\beta)/2} \psi^{-6+2\beta}\nuobs^\beta,  & \text{(off-axis)} \\
				\tobs^{(3+s_2)/4+3(-s_1+\beta)/8} \nuobs^\beta, &\text{(on-axis, pre jet break)} \\
				\tobs^{s_2/4 + 3(-s_1+\beta)/4} \nuobs^\beta, & \text{(on-axis, post jet break)} \end{matrix} \right . \ .
\end{equation}
These formulae capture the standard behavior of top-hat jets, as well as any jet that is \emph{fully} on-axis or off-axis.  What they fail to (easily) demonstrate is the behavior of a structured jet which transitions continuously from one state to the other over the course of observation.

\section{Derivation of the Structured Jet equations}\label{sec:derive2}

A jet with a non-trivial angular distribution of energy can exhibit qualitatively different behaviour than a simple top hat, particularly when observed at a significant viewing angle.  While the initial off-axis and final on-axis post jet-break evolutions are identical, these are separated by a transition phase where the sector dominating the emission scans over the jet surface.  This transition phase begins at the end of the off-axis phase: when a sector of the jet first decelerates to include the observer in its beaming cone.  This will necessarily be from the wings/edge of the jet, the material with the lowest Lorentz factor and smallest angle $\psi$ to the observer.  As the blastwave decelerates more energetic material from nearer the core will come into view and the high latitude emission will dim.  Finally the core of the jet ($\theta = 0$ or $\psi = \thobs$) decelerates and becomes visible to the observer.  At this point the entire jet is on-axis and evolution continues as in the post jet-break phase.

At each moment during the structure phase the emission is dominated by material that just came on-axis, where $\gamma^{-1} = \sin \psi$.  To find the overall behaviour we first determine $\tobs(\psi)$ and $F_\nu(\psi)$ for material whose emission is peaking (coming on-axis).  Since the structure phase occurs when motion is still relativistic, we can assume $\gamma \gg 1$ and hence $\sin \psi \ll 1$.  In this approximation:
\begin{eqnarray}
	\mu \approx 1 - \frac{1}{2}\sin^2\psi \ , \label{eq:muPsi}\\
	\delta_s \approx \delta \approx \gamma = \csc \psi\ . \label{eq:deltaPsi}
\end{eqnarray}
The material dominating the emission is in the plane between the observer and the jet axis, denoted by $\phi = 0$. Along this line we have $\psi = \thobs - \theta$, and can use $\psi$ or $\theta$ interchangeably to denote latitude.  From Equation \eqref{eq:lorentzEvolution} we have $t\propto E(\theta)^{1/3} \gamma^{-2/3}$.  Using Equation \eqref{eq:tobs} and \eqref{eq:muPsi} we find that material at $\psi$ will come on-axis at observer time:
\begin{equation}
	\tobs(\psi) = \frac{9}{16} t \sin^2\psi  \propto E(\theta)^{1/3} \sin^{8/3} \psi\ . \label{eq:tobsPsi}
\end{equation}
The peak flux from material at $\psi$ can be determined from Equation \eqref{eq:fluxApprox}.  Using Equation \eqref{eq:deltaPsi} and taking $\Delta \Omega \propto \gamma^{-s_3}$ gives an observed flux:
\begin{equation}
	F_\nu(\psi) \propto t^{3+s_2} \gamma^{2+s_1-s_3-\beta} \nuobs^\beta \propto E(\theta)^{1+s_2/3} (\sin \psi)^{-s_1 + 2 s_2/3 +s_3+\beta} \nuobs^\beta  \ . \label{eq:FnuPsi}
\end{equation}
	Equations \eqref{eq:tobsPsi} and \eqref{eq:FnuPsi} describe the evolution of the flux in the structure phase in terms of the parameter $\psi$, which varies from $\mathrm{min} (0, \thobs-\thW)$ to $\thobs$.  In principle one would like to invert Equation \eqref{eq:tobsPsi} and substitute into Equation \eqref{eq:FnuPsi} to obtain $F_\nu(\tobs)$ itself.  Unfortunately, in general this is impossible to do in closed form because $E(\theta)$ is non-trivial. 
	
	We can obtain the temporal power law slope of the light curve by differentiating both Equations \eqref{eq:tobsPsi} and \eqref{eq:FnuPsi} with respect to $\psi$.  Noting that $dE/d\psi = -dE/d\theta$ we obtain for the individual derivatives:
\begin{eqnarray}
	\frac{d \log \tobs}{d \psi} = \frac{8}{3} \cot \psi - \frac{1}{3} \frac{d \log E}{d \theta}\ , \\
	\frac{d \log F_\nu}{d \psi} = \left(-s_1 + \frac{2}{3} s_2 +s_3+\beta\right)\cot \psi - \left(1+\frac{1}{3}s_2\right) \frac{d \log E}{d \theta}\ .
\end{eqnarray}
Taking the ratio and simplifying gives:
\begin{eqnarray}
	\frac{d \log F_\nu}{d \log \tobs}(\psi) = \frac{3 \beta - 3s_1 + 2s_2+3s_3 + (3+s_2)g(\psi)}{ 8+g(\psi)} \\
	g(\psi) \equiv -\tan \psi \frac{d \log E}{d \theta}\ .
\end{eqnarray}
The parameter $g$ is directly measurable from the light curve, given the spectral information which fixes $\beta$, $s_1$, and $s_2$.

The full flux scaling equations also require an updated energy and circumburst density scaling. For energy, it follows that (below $\nu_c$)
\begin{equation}
	\frac{d \log F}{d \log E} = \frac{d \log F}{d \Psi} \frac{d \Psi}{d \log E} = - \frac{d \log F}{d \Psi} \frac{d \theta}{d \log E} = \left(s_1 - \frac{2}{3} s_2 - s_3 -\beta\right) h(\Psi) + 1 + \frac{1}{3} s_2,
\end{equation}
where
\begin{equation}
	h(\Psi) \equiv \cot \Psi \frac{d \log E}{d \theta}.
\end{equation}
For the density $n_0$ we have (below $\nu_c$)
\begin{equation}
	\frac{d \log F}{d \log n_0} = \frac{2+s_3}{8}
\end{equation}

Altogether, we obtain
\begin{eqnarray}
F_D & \propto & \epsilon_e^{-2/3} \epsilon_B^{1/3} n_0^{(2+s_3)/8} E^{\left(2/3 - s_3\right) h(\Psi) + 1} t_{obs}^{[-2+3s_3+3g(\Psi)]/[8+g(\Psi)]} \nu^{1/3} \nonumber \\
F_E & \propto & \epsilon_B^{1} n_0^{} \nonumber \\
F_G & \propto & \epsilon_e^{p-1} \epsilon_B^{(1+p)/4} n_0^{(2+s_3)/8} E^{2ph(\Psi)+1} t_{obs}^{3[-2p+s_3+g(\Psi)]/[8+g(\Psi)]} \nu^{(1-p)/2} \nonumber \\
F_H & \propto & \epsilon_e^{p-1} \epsilon_B^{(p-2)/4} n_0^{(-2+s_3)/8} E^{(2p + 2/3 - s_3)h(\Psi) + 2/3} t_{obs}^{[-6p-2+3s_3+2g(\Psi)]/[8+g(\Psi)]} \nu^{-p/2}
\end{eqnarray}

\section{Shock Jump Conditions and Synchrotron Parameters}
\label{sec:shockJump}

Table \ref{tab:shockSync} lists the necessary quantities for calculating synchrotron emission from a forward shock.  Asymptotic expressions are given in the ultra-relativistic and non-relativistic limits.  The synchrotron emissivity in the fluid rest frame is:
\begin{equation}
	\varepsilon'_{\nu'} = \varepsilon_P \times \left \{ \begin{matrix}
											\left(\nu' / \nu_m\right)^{1/3} & \text{if } \nu' < \nu_m < \nu_c \\
											\left(\nu' / \nu_m\right)^{(1-p)/2} & \text{if } \nu_m < \nu'  < \nu_c \\
											\left(\nu_c/\nu_m\right)^{(1-p)/2}\left(\nu' / \nu_c\right)^{-p/2} & \text{if }\nu_m < \nu_c < \nu'\\
											\left(\nu' / \nu_c\right)^{1/3} & \text{if } \nu' < \nu_c < \nu_m \\
											\left(\nu' / \nu_c\right)^{-1/2} & \text{if } \nu_c <\nu' <  \nu_m \\
											\left(\nu_m / \nu_c\right)^{-1/2}\left(\nu' / \nu_m\right)^{-p/2} & \text{if } \nu_c < \nu_m < \nu'
											\end{matrix} \right .
\end{equation}


\begin{deluxetable}{CC|CCC}
\tablehead{ & \colhead{Expression} & \colhead{Coefficient} & \colhead{$ u \gg 1$} & \colhead{$ u \ll 1$}}
\tablecaption{Emission parameters at forward shock. \label{tab:shockSync}}
\startdata
n & 4 \gamma n_0 & 4 n_0 & \gamma & 1 \\
e & (\gamma-1)n \Mp c^2 & 4 \Mp c^2 n_0 &  \gamma^2 & \frac{1}{2} \beta^2 \\
B & \sqrt{8\pi e \epsB} & 4(2 \pi)^{1/2}(\Mp c^2)^{1/2} n_0^{1/2} \epsB^{1/2} & \gamma & 2^{-1/2} \beta \\ 
\hline
\gamma_m &  \frac{\epseb e}{n \Me c^2} & \frac{\Mp}{\Me} \epseb &  \gamma & \frac{1}{2} \beta^2 \\
\gamma_c & \frac{3 \Me c \gamma}{4 \sigT \epsB e t} & \frac{3}{16} \frac{\Me}{\Mp\sigT c}n_0 ^{-1} \epsB^{-1} & \gamma^{-1} t^{-1} & 2 \beta^{-2} t^{-1} \\
\varepsilon_P & \frac{\sqrt{3}}{2}\frac{\qe^3}{\Me c^2}(p-1)n B & 8 (6\pi)^{1/2} \frac{\Mp^{1/2} \qe^3}{\Me c} (p-1) n_0^{3/2} \epsB^{1/2} & \gamma^2 & 2^{-1/2} \beta \\ 
\nu_m & \frac{3}{4\pi}\frac{\qe}{\Me c} \gamma_m^2 B & 6 (2\pi)^{-1/2} \frac{\Mp^{5/2} \qe}{\Me^3} n_0^{1/2} \epseb^2 \epsB^{1/2} & \gamma^3 & 2^{-5/2} \beta^5 \\
\nu_c & \frac{3}{4\pi}\frac{\qe}{\Me c} \gamma_c^2 B & \frac{27}{128} (2\pi)^{-1/2} \frac{\Me \qe}{\Mp^{3/2}\sigT^2 c^2} n_0^{-3/2}  \epsB^{-3/2} & \gamma^{-1}t^{-2} & 2^{3/2}\beta^{-3}t^{-2} \\
\hline
E & \frac{4\pi}{9} \rho_0 c^2 R^3(4u^2+3)\beta^2 & \frac{16\pi}{9}\rho_0 c^2 & R^3 \gamma^2 & \frac{3}{4} R^3 \beta^2 \\
\beta_s & \frac{4 u \gamma}{4u^2+3} & 1 & 1 & \frac{4}{3}\beta 
\enddata
\end{deluxetable}

\section{The Jet Angular Profile}
\label{sec:jetAngularShape}

The angular structure of a jet, before deceleration, can be described by angular profiles of energy $\epsilon(\theta, \phi)$ and Lorentz factor $\Gamma(\theta, \phi)$. This structure is acquired by the jet at its initial formation at the central engine and then sculpted by its propagation through the merger ejecta.  Due to the complications of direct numerical simulation much theoretical work has focused on simple analytic prescriptions of jet structure, typically axisymmetric uniform top-hat, Gaussian, or power law (CITE Granot+02, G+K03, K+G03, Rossi+04).

We consider the latter two models, a Gaussian profile
\begin{equation}
	E = E_0 \exp\left( \frac{\theta^2 }{2 \theta_c^2}\right)\ , \label{eq:Gaussian}
\end{equation}
and a $\theta^{-2}$ power law
\begin{equation}
	E = \frac{E_0}{1 + \theta^2/\theta_c^2}\ . \label{eq:powerlaw}
\end{equation}
Both models are truncated at an angle $\theta_w$. It is assumed all observed emission occurs after the jet has entered the deceleration phase, so the Lorentz factor profile is simply $\Gamma \propto E^{1/2}$.

With recent progress in hydrodynamical codes and computer resources, numerical simulation of jet propagation from the central engine to the circumburst medium has become possible.  Figure \ref{fig:jetStructure} shows energy and Lorentz factor distributions from hydrodynamic simulations collected from the recent literature, with representative examples of the Gaussian and power law models. The Lorentz factor distributions show large variation, in part due to differing times of extraction from the numerical simulations.  We focus comparison on the energy profile, as this is largely fixed while the jet remains relativistic. 

\begin{figure}
	\plotone{figs/jetStructure.pdf}
   	\figcaption{Energy and Lorentz factor profiles of jet models.  Profiles of numerical simulations are reported in Morsony+2007 (green), Mizuta+2009 (brown), Duffell+2015 (orange), Lazzati+2017 (pink), Kathirgamaraju+2017 (red), and Margutti+2018 (purple). The analytic boosted fireball model (Duffell+2015) is shown in blue.  Energy profiles for Morsony07, Mizuta09, and Duffell15 have been scaled to $dE/d\Omega(\theta=0) = 10^{52}$ erg/sr.  For comparison we also show representative power law ( Eq. \eqref{eq:powerlaw}, $\theta_c=1^\circ$, $\theta_w=60^\circ$, thick grey) and Gaussian (\eqref{eq:Gaussian}, $\theta_c=5.7^\circ$, $\theta_w=23^\circ$, thick black) profiles.  \label{fig:jetStructure}}
\end{figure}

There is agreement between the models that the jet energy is concentrated in a core of half-width $\lesssim 10^\circ$ with wings extending to larger angles.  Some models truncate at $\mathcal{O}(10)$ degrees, while others have large tails extending to near the equator.  The Gaussian and power-law models respectively emulate this behavior, providing a simple parameterized representation of the landscape of numerical models.  Of particular note are the Lazzati 2017 and Margutti 2018 models, both of which were calculated specifically for GW170817. The core of the Margutti model is has a profile very near Gaussian, while the Lazzati model has a strong resemblance to a power law.  



\end{document}
